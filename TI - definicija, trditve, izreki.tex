\documentclass[11pt]{article}
\usepackage[utf8]{inputenc}
\usepackage[slovene]{babel}

\usepackage{amsthm}
\usepackage{amsmath, amssymb, amsfonts}
\usepackage{relsize}
\usepackage{mathrsfs}

\newcommand{\R}{\mathbb{R}}
\newcommand{\N}{\mathbb{N}}
\newcommand{\A}{\mathcal{A}}
\newcommand{\diff}{\overset{\text{def}}{\iff}}
\newcommand{\set}[1]{\{#1\}}

\theoremstyle{definition}
\newtheorem{definicija}{Definicija}[section]

\theoremstyle{definition}
\newtheorem{trditev}{Trditev}[section]

\theoremstyle{definition}
\newtheorem{izrek}{Izrek}[section]

\theoremstyle{definition}
\newtheorem{metoda}{Metoda}[section]

\newtheorem*{posledica}{Posledica}
\newtheorem*{opomba}{Opomba}
\newtheorem*{komentar}{Komentar}
\newtheorem{lema}{Lema}
\newtheorem*{notacija}{Notacija}
\newtheorem*{dokaz}{Dokaz}
\newtheorem*{posplošitev}{Posplošitev}
\newtheorem*{dogovor}{Dogovor}
\newtheorem*{sklep}{Sklep}

\title{Teorija iger - definicije, trditve in izreki}
\author{Oskar Vavtar \\
po predavanjih profesorja Sergia Cabella}
\date{2021/22}

\begin{document}
\maketitle
\pagebreak
\tableofcontents
\pagebreak

% #################################################################################################

\section{Strateške igre s funkcijami preferenc}
\vspace{0.5cm}

% *************************************************************************************************

\subsection{Uvod}
\vspace{0.5cm}

\begin{definicija}

Naj bo $\A$ množica. \textit{Funkcija preferenc} na množici $\A$ je preslikava $u: \A \rightarrow \R$ funkcija preferenc. Intuicija: $\forall a,a' \in \A, a \neq a'$:
\begin{itemize}
	\item $u(a)>u(a')$ ``$\iff$'' $a$ je boljše kot $a'$
	\item $u(a)<u(a')$ ``$\iff$'' $a$ je slabše kot $a'$
	\item $u(a)=u(a')$ ``$\iff$'' med $a$ in $a'$ smo indiferentni
\end{itemize}

\end{definicija}
\vspace{0.5cm}

\begin{opomba}
~
\begin{itemize}
	\item Različne funkcije lahko določijo iste preference.
	\item Obravnavali bomo tudi več funkcij preferenc na isti množici (vsak igralec ima lahko svojo funkcijo).
	\item Preverence določimo kvalitativno, ne kvantitativno - pomemben je le vrstni red, same vrednosti ne.
	\item Namesto $\R$ bi lahko uporabili poljubno drugo linearno urejeno množico.
\end{itemize}

\end{opomba} 
\vspace{0.5cm}

\begin{definicija}

\textit{Strateška igra s funkcijami preferenc} je trojica
$$(N, (A_i)_{i \in N}, (u_i)_{i \in N}),$$
pri čemer:
\begin{itemize}
	\item $N$ je končna množica \textit{igralcev}.
	\item Za vsakega igralca $i \in N$ je $A_i$ neprazna množica \textit{akcij} za $i \in N$. Naj bo 
	$$\A ~:=~ \prod_{i \in N} A_i$$
	množica \textit{profilov akcij}.
	\item Za vsakega igralca $i \in N$ je $u_i: \A \rightarrow \R$ je funkcija preferenc na $\A$ za igralca $i$.
\end{itemize}

\end{definicija}
\vspace{0.5cm}

\begin{opomba}

Ponavadi: $N = [n] = \set{1,\ldots,n}$. V tem primeru imamo trojico
$$([n],(A_1,\ldots,A_n),(u_1,\ldots,u_n)),$$
$\A=A_1 \times \ldots \times A_n$ ter $u_i: A_1 \times \ldots \times A_n \rightarrow \R$.

\end{opomba}
\vspace{0.5cm}

% *************************************************************************************************

\subsection{Čisto Nashevo ravnotežje}
\vspace{0.5cm}

\begin{notacija}

$$(x_\alpha, x_\beta, x_\gamma \mid y; \beta) ~=~ (x_{-\beta}, y) ~=~ (x_\alpha, y, x_\gamma)$$
Za funkcije preferenc:
$$u_i(x_1,\ldots,x_m \mid y) ~=~ u_i(x_1,\ldots,x_{i-1},y,x_{i+1},\ldots,x_n).$$

\end{notacija}
\vspace{0.5cm}

\begin{definicija}

Naj bo $\Gamma = (N, (A_i)_{i \in N}, (u_i)_{i \in N})$ strateška igra s funkcijami preferenc. Naj bo
$$\A ~=~ \prod_{i \in N} A_i.$$
Profil akcij $a^* \in \A$ je \textit{čisto Nashevo ravnovesje} $\diff$
$$\forall i \in N, ~\forall b \in A_i: ~u_i(a^*) \geq u_i(a^* \mid b).$$
Tak $a^* \in \A$ je \textit{strogo čisto Nashevo ravnovesje} $\diff$
$$\forall i \in N, ~\forall b \in A_i\setminus\set{a_i^*}: ~u_i(a^*) > u_i(a^* \mid b).$$

\end{definicija}
\vspace{0.5cm}

\pagebreak

\begin{definicija}

Naj bo $\Gamma = (N, (A_i)_{i \in N}, (u_i)_{i \in N})$ strateška igra s funkcijami preferenc. Označimo
$$\A ~=~ \prod_{n \in N} A_i.$$
\textit{Najboljši odgovor} igralca $i \in N$ je 
\begin{align*}
B_i: ~\A ~&\rightarrow~ 2^{A_i} ~=~ \set{B \mid B \subseteq A_i} \\
a ~&\mapsto~ \set{b \in A_i \mid \forall c \in A_i: u_i(a \mid b) \geq u_i(a \mid c)} \\
&=~ \set{b \in A_i \mid u_i(a \mid b) ~=~ \max_{c \in A_i}(a \mid c)}.
\end{align*}
Za $N = [n] = \set{1,\ldots,n}$ je formula v definiciji za igralca $i=1:$
$$B_1(a_1,\ldots,a_n) ~=~ \set{b \in A_1 \mid u_1(b,a_2,\ldots,a_n) = \max_{c \in A_1} u_1(c,a_2,\ldots,a_n)},$$
analogno za $i=2,\ldots,n$.

\end{definicija}
\vspace{0.5cm}

\begin{opomba}
~
\begin{itemize}
	\item Bolj pravilno bi bilo reči ``\textit{množica najboljših odgovorov}''.
	\item Večkrat velja $|B_i(a)|=1$ (le en najboljši odgovor). V tem primeru pišemo brez $\set$.
	\item Pri definiciji $B_i$ nima $a_i$ nobene vloge. Za dva igralca bomo ponavadi napisali
	\begin{align*}
	B_1(a_2) ~&\equiv~ B_1(*,a_2) \\
	B_2(a_1) ~&\equiv~ B_2(a_1,*).
	\end{align*}
\end{itemize}

\end{opomba} 
\vspace{0.5cm}

\begin{trditev}

Profil akcij $a^* = (a_i^*)_{i \in N}$ je čisto Nashevo ravnovesje $\iff$
$$\forall i \in N: ~a_i^* \in B_i(a^*).$$

\end{trditev}
\vspace{0.5cm}

\pagebreak

\begin{trditev}
Profil akcij $a^* = (a_i^*)_{i \in N}$ je strogo čisto Nashevo \hbox{ravnovesje $\iff$}
$$\forall i \in N: ~a_i^* \in B_i^\text{str}(a^*),$$
kjer $B_i^\text{str}$ definiramo kot
\begin{align*}
B_i^\text{str}: ~\A ~&\rightarrow~ 2^{A_i} \\
a ~&\mapsto~ \set{b \in A_i \mid \forall c \in A_i \setminus \set{b}: u_i(a^* \mid b) > u_i(a^* \mid c)} \\
&=~ \begin{cases}
\text{edini max}; ~&\text{če obstaja}, \\
\emptyset; ~&\text{sicer}.
\end{cases}
\end{align*}
\end{trditev}
\vspace{0.5cm}

% *************************************************************************************************

\subsection{Dominacije}
\vspace{0.5cm}

\begin{definicija}

Naj bo $\Gamma = (N, (A_i)_{i \in N}, (u_i)_{i \in N})$ strateška igra s funkcijo preferenc. Označimo
$$\A ~=~ \prod_{i \in N} A_i.$$
Akcija $b \in A_i$ \textit{šibko dominira} akcijo $c \in A_i$, če velja
$$\forall a \in \A: ~u_i(a \mid b) \geq u_i(a \mid c).$$
Akcija $b \in A_i$ \textit{strogo dominira} akcijo $c \in A_i$, če velja
$$\forall a \in \A: ~u_i(a \mid b) > u_i(a \mid c).$$

\end{definicija}
\vspace{0.5cm}

\begin{trditev}

Če $b \in A_i$ strogo dominira $c \in A_i$, potem ne obstaja čisto Nashevo ravnovesje $a^* = (a_i^*)_{i \in N}$ z $a_i^* = c$. Če $b \in A_i$ dominira $c \in A_i$, potem obstaja strogo čisto Nashevo ravnovesje $a^* = (a_i^*)_{i \in N}$ z $a_i^* = c$.

\end{trditev}
\vspace{0.5cm}

% *************************************************************************************************

\pagebreak

% #################################################################################################

\end{document}