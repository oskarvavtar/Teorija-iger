\documentclass[11pt]{article}
\usepackage[utf8]{inputenc}
\usepackage[slovene]{babel}

\usepackage{amsthm}
\usepackage{amsmath, amssymb, amsfonts}
\usepackage{relsize}
\usepackage{mathrsfs}
\usepackage{bbm}

\newcommand{\R}{\mathbb{R}}
\newcommand{\N}{\mathbb{N}}
\newcommand{\A}{\mathscr{A}}
\renewcommand{\S}{\mathscr{S}}
\newcommand{\diff}{\overset{\text{def}}{\iff}}
\newcommand{\set}[1]{\{#1\}}
\newcommand{\oklepaj}[1]{\left(#1\right)}
\newcommand{\oglati}[1]{\left[#1\right]}
\renewcommand{\P}{\mathcal{P}}
\renewcommand{\O}{\mathcal{O}}
\newcommand{\T}{\textsf{T}}
\newcommand{\TT}{\mathcal{T}}
\newcommand{\V}{\mathscr{V}}
\renewcommand{\L}{\mathscr{L}}
\newcommand{\E}{\mathscr{E}}
\newcommand{\p}{\mathbf{p}}
\newcommand{\q}{\mathbf{q}}
\newcommand{\x}{\mathbf{x}}
\newcommand{\y}{\mathbf{y}}
\renewcommand{\b}{\mathbf{b}}
\renewcommand{\c}{\mathbf{c}}
\renewcommand{\AA}{\mathbf{A}}
\newcommand{\BB}{\mathbf{B}}
\newcommand{\1}{\mathbbm{1}}
\newcommand{\bay}{\mathscr{B}}
\newcommand{\kor}{\mathscr{K}}
\newcommand{\eks}{\mathscr{E}}
\newcommand{\pref}{\mathscr{P}}
\renewcommand{\N}{\mathcal{N}}
\newcommand{\D}{\mathcal{D}}

\theoremstyle{definition}
\newtheorem{definicija}{Definicija}[section]

\theoremstyle{definition}
\newtheorem{trditev}{Trditev}[section]

\theoremstyle{definition}
\newtheorem{izrek}{Izrek}[section]

\theoremstyle{definition}
\newtheorem{metoda}{Metoda}[section]

\newtheorem*{posledica}{Posledica}
\newtheorem*{opomba}{Opomba}
\newtheorem*{komentar}{Komentar}
\newtheorem{lema}{Lema}
\newtheorem*{notacija}{Notacija}
\newtheorem*{dokaz}{Dokaz}
\newtheorem*{posplošitev}{Posplošitev}
\newtheorem*{dogovor}{Dogovor}
\newtheorem*{sklep}{Sklep}

\title{Teorija iger - definicije, trditve in izreki}
\author{Oskar Vavtar \\
po predavanjih profesorja Sergia Cabella}
\date{2021/22}

\begin{document}
\maketitle
\pagebreak
\tableofcontents
\pagebreak

% #################################################################################################

\section{Strateške igre s funkcijami preferenc}
\vspace{0.5cm}

% *************************************************************************************************

\subsection{Uvod}
\vspace{0.5cm}

\begin{definicija}

Naj bo $\A$ množica. \textit{Funkcija preferenc} na množici $\A$ je preslikava $u: \A \rightarrow \R$ funkcija preferenc. Intuicija: $\forall a,a' \in \A, a \neq a'$:
\begin{itemize}
	\item $u(a)>u(a')$ ``$\iff$'' $a$ je boljše kot $a'$
	\item $u(a)<u(a')$ ``$\iff$'' $a$ je slabše kot $a'$
	\item $u(a)=u(a')$ ``$\iff$'' med $a$ in $a'$ smo indiferentni
\end{itemize}

\end{definicija}
\vspace{0.5cm}

\begin{opomba}
~
\begin{itemize}
	\item Različne funkcije lahko določijo iste preference.
	\item Obravnavali bomo tudi več funkcij preferenc na isti množici (vsak igralec ima lahko svojo funkcijo).
	\item Preverence določimo kvalitativno, ne kvantitativno - pomemben je le vrstni red, same vrednosti ne.
	\item Namesto $\R$ bi lahko uporabili poljubno drugo linearno urejeno množico.
\end{itemize}

\end{opomba} 
\vspace{0.5cm}

\begin{definicija}

\textit{Strateška igra s funkcijami preferenc} je trojica
$$(N, (A_i)_{i \in N}, (u_i)_{i \in N}),$$
pri čemer:
\begin{itemize}
	\item $N$ je končna množica \textit{igralcev}.
	\item Za vsakega igralca $i \in N$ je $A_i$ neprazna množica \textit{akcij} za $i \in N$. Naj bo 
	$$\A ~:=~ \prod_{i \in N} A_i$$
	množica \textit{profilov akcij}.
	\item Za vsakega igralca $i \in N$ je $u_i: \A \rightarrow \R$ je funkcija preferenc na $\A$ za igralca $i$.
\end{itemize}

\end{definicija}
\vspace{0.5cm}

\begin{opomba}

Ponavadi: $N = [n] = \set{1,\ldots,n}$. V tem primeru imamo trojico
$$([n],(A_1,\ldots,A_n),(u_1,\ldots,u_n)),$$
$\A=A_1 \times \ldots \times A_n$ ter $u_i: A_1 \times \ldots \times A_n \rightarrow \R$.

\end{opomba}
\vspace{0.5cm}

% *************************************************************************************************

\subsection{Čisto Nashevo ravnotežje}
\vspace{0.5cm}

\begin{notacija}

$$(x_\alpha, x_\beta, x_\gamma \mid y; \beta) ~=~ (x_{-\beta}, y) ~=~ (x_\alpha, y, x_\gamma)$$
Za funkcije preferenc:
$$u_i(x_1,\ldots,x_m \mid y) ~=~ u_i(x_1,\ldots,x_{i-1},y,x_{i+1},\ldots,x_n).$$

\end{notacija}
\vspace{0.5cm}

\begin{definicija}

Naj bo $\Gamma = (N, (A_i)_{i \in N}, (u_i)_{i \in N})$ strateška igra s funkcijami preferenc. Naj bo
$$\A ~=~ \prod_{i \in N} A_i.$$
Profil akcij $a^* \in \A$ je \textit{čisto Nashevo ravnovesje} $\diff$
$$\forall i \in N, ~\forall b \in A_i: ~u_i(a^*) \geq u_i(a^* \mid b).$$
Tak $a^* \in \A$ je \textit{strogo čisto Nashevo ravnovesje} $\diff$
$$\forall i \in N, ~\forall b \in A_i\setminus\set{a_i^*}: ~u_i(a^*) > u_i(a^* \mid b).$$

\end{definicija}
\vspace{0.5cm}

\pagebreak

\begin{definicija}

Naj bo $\Gamma = (N, (A_i)_{i \in N}, (u_i)_{i \in N})$ strateška igra s funkcijami preferenc. Označimo
$$\A ~=~ \prod_{n \in N} A_i.$$
\textit{Najboljši odgovor} igralca $i \in N$ je 
\begin{align*}
B_i: ~\A ~&\rightarrow~ 2^{A_i} ~=~ \set{B \mid B \subseteq A_i} \\
a ~&\mapsto~ \set{b \in A_i \mid \forall c \in A_i: u_i(a \mid b) \geq u_i(a \mid c)} \\
&=~ \set{b \in A_i \mid u_i(a \mid b) ~=~ \max_{c \in A_i}(a \mid c)}.
\end{align*}
Za $N = [n] = \set{1,\ldots,n}$ je formula v definiciji za igralca $i=1:$
$$B_1(a_1,\ldots,a_n) ~=~ \set{b \in A_1 \mid u_1(b,a_2,\ldots,a_n) = \max_{c \in A_1} u_1(c,a_2,\ldots,a_n)},$$
analogno za $i=2,\ldots,n$.

\end{definicija}
\vspace{0.5cm}

\begin{opomba}
~
\begin{itemize}
	\item Bolj pravilno bi bilo reči ``\textit{množica najboljših odgovorov}''.
	\item Večkrat velja $|B_i(a)|=1$ (le en najboljši odgovor). V tem primeru pišemo brez $\set$.
	\item Pri definiciji $B_i$ nima $a_i$ nobene vloge. Za dva igralca bomo ponavadi napisali
	\begin{align*}
	B_1(a_2) ~&\equiv~ B_1(\cdot,a_2) \\
	B_2(a_1) ~&\equiv~ B_2(a_1,\cdot).
	\end{align*}
\end{itemize}

\end{opomba} 
\vspace{0.5cm}

\begin{trditev}

Profil akcij $a^* = (a_i^*)_{i \in N}$ je čisto Nashevo ravnovesje $\iff$
$$\forall i \in N: ~a_i^* \in B_i(a^*).$$

\end{trditev}
\vspace{0.5cm}

\pagebreak

\begin{trditev}
Profil akcij $a^* = (a_i^*)_{i \in N}$ je strogo čisto Nashevo \hbox{ravnovesje $\iff$}
$$\forall i \in N: ~a_i^* \in B_i^\text{str}(a^*),$$
kjer $B_i^\text{str}$ definiramo kot
\begin{align*}
B_i^\text{str}: ~\A ~&\rightarrow~ 2^{A_i} \\
a ~&\mapsto~ \set{b \in A_i \mid \forall c \in A_i \setminus \set{b}: u_i(a^* \mid b) > u_i(a^* \mid c)} \\
&=~ \begin{cases}
\text{edini max}; ~&\text{če obstaja}, \\
\emptyset; ~&\text{sicer}.
\end{cases}
\end{align*}
\end{trditev}
\vspace{0.5cm}

% *************************************************************************************************

\subsection{Dominacije}
\vspace{0.5cm}

\begin{definicija}

Naj bo $\Gamma = (N, (A_i)_{i \in N}, (u_i)_{i \in N})$ strateška igra s funkcijo preferenc. Označimo
$$\A ~=~ \prod_{i \in N} A_i.$$
Akcija $b \in A_i$ \textit{šibko dominira} akcijo $c \in A_i$, če velja
$$\forall a \in \A: ~u_i(a \mid b) \geq u_i(a \mid c).$$
Akcija $b \in A_i$ \textit{strogo dominira} akcijo $c \in A_i$, če velja
$$\forall a \in \A: ~u_i(a \mid b) > u_i(a \mid c).$$

\end{definicija}
\vspace{0.5cm}

\begin{trditev}

Če $b \in A_i$ strogo dominira $c \in A_i$, potem ne obstaja čisto Nashevo ravnovesje $a^* = (a_i^*)_{i \in N}$ z $a_i^* = c$. Če $b \in A_i$ dominira $c \in A_i$, potem obstaja strogo čisto Nashevo ravnovesje $a^* = (a_i^*)_{i \in N}$ z $a_i^* = c$.

\end{trditev}
\vspace{0.5cm}

% *************************************************************************************************

\pagebreak

% #################################################################################################

\section{Strateške igre s funkcijami koristnosti}
\vspace{0.5cm}

% *************************************************************************************************

\subsection{Uvod}
\vspace{0.5cm}

\begin{definicija}

Naj bo $A = (a_1,\ldots,a_\Pi)$ končna množica in $\pi$ funkcija verjetnosti:
$$\pi ~\sim~ \begin{pmatrix}
a_1 & \ldots & a_\Pi \\
\pi(a_1) & \ldots & \pi(a_\Pi)
\end{pmatrix}.$$ 
Množica $\pi(A)$, definirana kot
$$\pi(A) ~=~ \set{((\pi(a_i))_{a_i \in A} \mid \forall a_i \in A: \pi(a_i) \geq 0, ~\sum_{a_i \in A} \pi(a_i) = 1}$$
je \textit{množica loterij} na $A$.

\end{definicija}
\vspace{0.5cm}

\begin{definicija}

Naj bo $A$ končna. \textit{Funkcija koristnosti} na $A$ je prelikava $u: A \rightarrow \R$, ki določa preference na množici $\pi(A)$ in sicer
\begin{align*}
\hat{u}: \pi(A) ~&\rightarrow~ \R \\
\pi = (\pi(a))_{a \in A} ~&\mapsto~ \sum_{a \in A} \pi(a) u(a),
\end{align*}
če je $\hat{u}$ razširitev funkcije $u$. Osnovni princip bo:
$$\hat{u}(\pi) \geq \hat{u}(\pi') ~~~\iff~~~ \pi ~\text{ni slabše od}~ \pi'.$$

\end{definicija}
\vspace{0.5cm}

\begin{opomba}

Lahko imamo različni funkciji koristnosti, ki določata iste preference na $\pi(A)$. En del teorije koristnosti se ukvaraja z obratno smerjo, in sicer za katere preference nad $\pi(A)$ obstaja funkcija koristnosti $u: A \rightarrow \R$, za katero imamo razširitev iste preference $\pi(A)$.

\end{opomba}
\vspace{0.5cm}

% *************************************************************************************************

\subsection{Igre koristnosti}
\vspace{0.5cm}

\begin{definicija}

\textit{Strateška igra s funkcijami koristnosti} je trojica 
$$\Gamma ~=~ (N, (A_i)_{i \in N}, (u_i)_{i \in N}),$$
pri čemer:
\begin{itemize}
	\item je $N$ neprazna in končna množica igralcev.
	\item je $A_i$ neprazna in končna množica akcij za igralca $i \in N$. Naj bo $\A = \prod_{i \in N} A_i$ množica profilov akcij.
	\item je $u_i: A \rightarrow \R$ funkcija koristnosti za igralca $i \in N$.
\end{itemize}

\noindent Množica strategij za igralca $i \in N$ je $S_i = \pi(A_i)$. Strategija $\pi \in S_i$ meša akcije $\set{a \in A_i \mid \pi_i(a) > 0}$. Strategija $\pi \in S_i$ je čista, če je $\pi = S(a)$ za nek $a \in A_i$. Množica profilov strategij je $\S = \prod_{i \in N} S_i$.

\end{definicija}
\vspace{0.5cm}

\begin{trditev}

Definiramo:
\begin{align*}
u_i: \S = \prod_{j \in N} S_j ~&\rightarrow~ \R \\
\pi = (\pi_j)_{j \in N} ~&\mapsto~ \sum_{a \in \A} \oklepaj{\prod_{j \in N}\pi_j(a_j)} u_i(a).
\end{align*}
Velja:
$$u_i(\pi) ~=~ \sum_{a_i \in A_i} \pi_i(a_i) \oglati{u_i(\pi \mid \delta(a_i)}.$$

\end{trditev}
\vspace{0.5cm}

% *************************************************************************************************

\subsection{Nasheva ravnovesja}
\vspace{0.5cm}

\begin{definicija}

Naj bo $\Gamma(N, (A_i)_{i \in N}, (u_i)_{i \in N})$ strateška igra s funkcijami koristnosti. Profil strategij 
$$\Pi^* = (\pi_i^*) ~\in~ \S = \prod_{i \in N} S_i$$
je Nashevo ravnovesje $\diff$
$$\forall i \in N, ~\forall \pi'_i \in S_i, ~\pi'_i \neq \pi_i^*: ~u_i^*(\pi^*) ~>~ u_i(\pi^* \mid \pi'_i).$$
Imamo preslikavo
\begin{align*}
\phi_{\mathscr{K} \rightarrow \mathscr{P}}: \set{\text{igra s funkcijami koristnosti}} ~&\rightarrow~ \set{\text{igra s funkcijami preferenc}} \\
(N, (A_i)_{i \in N}, (u_i)_{i \in N}) ~&\mapsto~ (N, (S_i = \pi(A_i))_{i \in N}, (u_i)_{i \in N}).
\end{align*}
$\phi_{\mathscr{K} \rightarrow \mathscr{P}}$ je \textit{mešana razširitev} igre $\Gamma$.

\end{definicija}
\vspace{0.5cm}

\begin{trditev}

Naj bo $\Gamma$ strateška igra s funkcijami koristnosti in naj bo $\pi$ profil strategij v $\Gamma$. $\pi$ je Nashevo ravnovesje v $\Gamma$ $\iff$ $\pi$ je čisto Nashevo ravnovesje v $\phi_{\mathscr{K} \rightarrow \mathscr{P}}(\Gamma)$.

\end{trditev}
\vspace{0.5cm}

\begin{izrek}[Nash, 1949]

Vsaka strateška igra s funkcijami koristnosti ima vsaj eno Nashevo ravnovesje.

\end{izrek}
\vspace{0.5cm}

\begin{opomba}
~
\begin{itemize}
	\item Dokaz je nekonstruktiven, nič ne pove kako Nashevo ravnovesje najti.
	\item Pomembno je, da ima vsak igralec končno množico akcij. Obstajajo sicer tudi posplošitve za neskončno mnogo akcij.
	\item Če je igra generična, potem je število Nashevih ravnovesij liho.
\end{itemize}

\end{opomba}
\vspace{0.5cm}

\begin{trditev}

Profil strategij $\pi^* = (\pi_i^*)_{i\in N}$ je Nashevo ravnovesje
$$\iff~ \forall i \in N ~\forall a_i \in A_i: ~u_i(\pi^*) \geq u_i(\pi^* \mid \delta(a_i)).$$

\end{trditev}
\vspace{0.5cm}

\begin{trditev}[Princip indiferentnosti]

Če je $\pi^* = (\pi_i^*)_{i \in N}$ Nashevo ravnovesje, potem velja
$$\forall i \in N ~\forall a_i \in A_i: ~(\pi_i^*(a_i) > 0 ~\Rightarrow~ u_i(\pi^*) = u_i(\pi^* \mid \delta(a))).$$

\end{trditev}
\vspace{0.5cm}

\begin{opomba}
~
\begin{itemize}
	\item Uporabna, ker imamo enačbe.
	\item To ni karakterizacija. Uporabimo za iskanje kandidatov.
	\item Ko ima igralec $i$ $|A_i| = 2$, potem, ko velja
	$$u_i(\pi^* \mid \delta(a_i)) ~=~ u_i(\pi^* \mid \delta(b_i)) ~=~ u_i(\pi^*)$$
	avtomatično velja pogoj sistema neenačb za igralca $i$:
	\begin{align*}
	u_i(\pi^* \mid \delta(a_i)) ~\leq~ u_i(\pi^*) \\
	u_i(\pi^* \mid \delta(b_i)) ~\leq~ u_i(\pi^*)
	\end{align*}
\end{itemize}

\end{opomba}
\vspace{0.5cm}

% *************************************************************************************************

\subsection{Dominacije}
\vspace{0.5cm}

\begin{definicija}

Strategija $\alpha_i \in S_i$ \textit{(šibko) dominira} strategijo $\beta_i \in S_i$, če velja
$$\forall \pi \in S: ~u_i(\pi \mid \alpha_i) ~\geq~ u_i(\pi \mid \beta_i).$$
Strategija $\alpha_i \in S_i$ \textit{strogo dominira} $\beta_i \in S_i$, če velja
$$\forall \pi \in S: ~u_i(\pi \mid \alpha_i) ~>~ u_i(\pi \mid \beta_i).$$

\end{definicija}
\vspace{0.5cm}

\begin{trditev}

Naj bo $\pi^* = (\pi_i^*)_{i \in N}$ Nashevo ravnovesje. Če strategija \hbox{$\alpha_i \in S_i$} strogo dominira $\delta(b_i)$, potem 
$$\pi_i^*(b_i) ~=~ 0.$$

\end{trditev}
\vspace{0.5cm}

% *************************************************************************************************

\subsection{Dokaz Nashevega izreka \\(dokaz je v zapiskih, tu le potrebno ``orodje'')}
\vspace{0.5cm}

\begin{izrek}[Brouwerjev izrek]

Naj bo $X \subseteq \R^d$ konveksna in kompaktna množica ter $T: X \rightarrow X$ zvezna preslikava. Potem $\exists x \in X$, da velja \hbox{$T(x) = x$}.

\end{izrek}
\vspace{0.5cm}

\pagebreak

% #################################################################################################

\section{Bimatrične in matrične igre}
\vspace{0.5cm}

% *************************************************************************************************

\subsection{Uvod}
\vspace{0.5cm}

\begin{definicija}

\textit{Bimatrična igra} je igra s funkcijami koristnosti za 2 igralca, $N = \set{1,2}$, pri katerih je
\begin{align*}
A_1 ~&=~ [m] ~=~ \set{1,\ldots,m} \\
A_2 ~&=~ [n] ~=~ \set{1,\ldots,n}.
\end{align*}
Igro predstavimo z bimatriko.

\end{definicija}
\vspace{0.5cm}

\begin{trditev}

Profil $(\p^*,\q^*) \in \pi_1 \times \pi_2$ je Nashevo ravnovesje $\iff$
\begin{align*}
\forall \p \in \pi_1: ~(\p^*)^\T \AA \q^* ~&\geq~ \p^\T \AA \q^* \\
\forall \q \in \pi_2: ~(\p^*)^\T \BB \q^* ~&\geq~ (\p^*)^\T \BB \q
\end{align*}

\end{trditev}
\vspace{0.5cm}

\begin{trditev}[Sistem neenačb]

Profil $(\p^*,\q^*)$ je Nashevo ravnovesje $\iff$
\begin{align*}
\forall i \in [m]: ~(\p^*)^\T \AA \q^* ~&\geq~ [\AA\q^*]_i \\
\forall j \in [n]: ~(\p^*)^\T \BB \q^* ~&\geq~ [(\p^*)^\T \BB]_j
\end{align*}

\end{trditev}
\vspace{0.5cm}

\begin{trditev}[Princip indiferentnosti]

Če je profil $(\p^*, \q^*)$ Nashevo ravnovesje, potem
\begin{align*}
\forall i \in [m]: ~(p_i^* > 0) ~&\Rightarrow~ (\p^*)^\T \AA \q^* = [\AA\q^*]_i \\
\forall j \in [n]: ~(q_j^* > 0) ~&\Rightarrow~ (\p^*)^\T \AA \q^* = [(\p^*)^\T \AA]_j.
\end{align*}

\end{trditev}
\vspace{0.5cm}

% *************************************************************************************************

\subsection{Linearno programiranje}
\vspace{0.5cm}

\begin{definicija}

Linearni program izgleda tako:
\begin{align*}
\max \c^\T \x& \\
\text{p.p}~ \AA\x &\leq \b \\
\c,\x &\in \R^n \\
\AA &\in \R^{m \times n} \\
\b &\in \R^m \\
\x &\geq \mathbf{0} ~\text{ali}~ x_i \geq 0 ~\text{za nekatere}~ i
\end{align*}
Linearni program je lahko:
\begin{itemize}
	\item dopusten: $\exists \x \in \R^n: \AA\x \leq \b$
	\begin{itemize}
		\item neomejen: $\forall k \in \R~\exists \x \in \R^n: \c^\T\x \geq k,~\AA\x\leq\b$
		\item omejen: $\sup\set{\c^\T\x; ~\x\in\R^n,~\AA\x\leq\b}$ obstaja in je $\max$, torej $\exists \x^* \in \R^n: \c^\T\x^* = \sup{\c^\T\x; ~\x\in\R^n,\AA\x\leq\b}$, je optimalna rešitev, $\AA\x^*\leq\b$
	\end{itemize}
	\item nedopusten: $\set{\x \in \R^n; ~\AA\x \leq \b} = \emptyset$
\end{itemize}

\end{definicija}
\vspace{0.5cm}

\begin{definicija} Dualnost:
\begin{itemize}

\item Originalen problem:
\begin{align*}
\max ~&\c^\T\x \\
\text{p.p.}~ &\oglati{\AA\x}_{i = 1,\ldots,j} \leq \oglati{\b}_{i=1,\ldots,j} \\
&\oglati{\AA\x}_{i = j+1,\ldots,m} = \oglati{\b}_{i=j+1,\ldots,m} \\
&x_1,\ldots,x_k \geq 0 \\
&x_{k+1},\ldots,x_n \in \R
\end{align*}

\item Dualni problem:
\begin{align*}
\max ~&\b^\T\y \\
\text{p.p.}~ &\oglati{\AA^\T\y}_{i = 1,\ldots,k} \leq \oglati{\c}_{i=1,\ldots,k} \\
&\oglati{\AA^\T\y}_{i = k+1,\ldots,n} = \oglati{\c}_{i=k+1,\ldots,n} \\
&y_1,\ldots,y_j \geq 0 \\
&y_{j+1},\ldots,y_n \in \R
\end{align*}

\end{itemize}
\end{definicija}
\vspace{0.5cm}

\begin{izrek}

Če je originalni problem omejen in ima optimalno rešitev $\x^*$, potem je dualni problem omejen in za vsako rešitev dualnega problema $\y^*$ velja
$$\c^\T\x^* ~=~ \b^\T\y^*$$
(optimalni rešitvi sta isti, če je problem omejen.

\end{izrek}
\vspace{0.5cm}

% *************************************************************************************************

\subsection{Stopnja varnosti}
\vspace{0.5cm}

\begin{definicija}

Naj bo $\oglati{\AA,\BB}$ bimatrična igra. \textit{Stopnja varnosti} za 1. igralca je
$$V_1 ~=~ \max_{\p\in\Pi_1}\min_{\q\in\Pi_2} \p^\T\AA\q ~=~ \max_{\p\in\Pi_1}\min_{\q\in\Pi_2} U_1(\p,\q).$$
Maxmin strategija za 1. igralca je $\overline{\p}\in \Pi_1$, za katero velja $V_1 = \min_{\q\in\Pi_2}\overline{\p}^\T\AA\q$. Stopnja varnosti za 2. igralca je
$$V_2 ~=~ \max_{\q\in\Pi_2}\min_{\p\in\Pi_1}\p^\T\BB\q.$$
Maxmin strategija za 2. igralca je $\overline{\q}\in\Pi_2$, za katero je $V_2 = \min_{\p\in\Pi_1}\p^\T\BB\overline{q}$. Bolj splošno:
\begin{align*}
V_1 ~&=~ \max_{\p\in\Pi_1}\min_{\q\in\Pi_2}\sum_{j\in [n]}\oglati{\p^\T\AA}_j \q_j ~=~ \max_{\p\in\Pi_1}\min_{j\in [n]}\oglati{\p^\T\AA}_j \\
V_2 ~&=~ \max_{\q\in\Pi_2}\min_{i\in [m]}\oglati{\BB\q}_i
\end{align*}

\end{definicija}
\vspace{0.5cm}

\begin{trditev}

Če je $\p\in\Pi_1$ maxmin strategija, potem
$$\forall\q\in\Pi_2: ~V_1 \leq \p^\T\AA\q.$$

\end{trditev}
\vspace{0.5cm}

\begin{trditev}

$\set{\p\in\Pi_1; ~\p~\text{je maxmin}}$ in $\set{\q\in\Pi_2; ~\q~\text{je maxmin}}$ sta \hbox{konveksni.}

\end{trditev}
\vspace{0.5cm}

\begin{trditev}

Računanje stopnje varnosti je linearni program.

\end{trditev}
\vspace{0.5cm}

% *************************************************************************************************

\subsection{Matrične igre}
\vspace{0.5cm}

\begin{definicija}

\textit{Matrična igra} je bimatrična igra $\oklepaj{\AA,\BB}$ z $\AA=-\BB$. Torej:
\begin{itemize}
	\item 2 igralca
	\item 1. izbere $\AA_1 = [m]$
	\item 2. izbere $\AA_2 = [n]$
	\item $\AA_{i,j}=-\BB_{i,j}$ nam pove koliko 2. igralec plača 1. igralcu
	\item $\AA$ je izplačilna matrika
\end{itemize}

\end{definicija}
\vspace{0.5cm}

\begin{izrek}[von Neumann; minimax izrek]

Za vsako matrično igro velja $V_1 = -V_2$ oziroma
$$\max_\p \min_\q \p^\T\AA\q ~=~ \min_\q \max_\p \p^\T\AA\q.$$

\end{izrek}
\vspace{0.5cm}

\begin{opomba}
~
\begin{itemize}
	\item V spločnem $\max\min \neq \min\max$.
	\item Uporabili bomo dualnost.
\end{itemize}

\end{opomba}
\vspace{0.5cm}

\begin{posledica}

Če igralca uporabita svoje maxmin strategije $\p$ in $\q$, potem $V_1 = \p^\T\AA\q$.

\end{posledica}
\vspace{0.5cm}

\begin{posledica}
~
\begin{itemize}
	\item Če je $\p$ maxmin strategija 1. igralca, potem $\forall \q\in\Pi_2: \p^\T\AA\q \geq V_1$.
	\item Če je $\q$ maxmin strategija 2. igralca, potem $\forall \p\in\Pi_1: \p^\T\AA\q \leq V_1$. 
\end{itemize}
Vrednost igre je $V_1(\AA) =: V(\AA)$.

\end{posledica}
\vspace{0.5cm}

\begin{definicija}

Pravimo, da je igra \textit{poštena}, če $V(\AA)=0$.

\end{definicija}
\vspace{0.5cm}

\begin{posledica}

Naj bo $\AA$ matrična igra, $\p\in\Pi_1$, $\q\in\Pi_2$, $u\in\R$. Če velja $\p^\T\AA \geq u\cdot \1_n^\T$ in $\AA\q \leq u \cdot \1_m$, potem je $u=V(\AA)$ ter $\p,\q$ maxmin strategiji.

\end{posledica}
\vspace{0.5cm}

\begin{trditev}

Če je $(\p,\q)\in\Pi_1\times\Pi_2$ Nashevo ravnovesje, potem sta $\p$ in $\q$ maxmin strategiji.

\end{trditev}
\vspace{0.5cm}

\begin{trditev}

Če je $\p$ maxmin za 1. igralca in $\q$ maxmin za 2. igralca, potem je $(\p,\q)$ Nashevo ravnovesje.

\end{trditev}
\vspace{0.5cm}

\begin{izrek}

$(\p,\q)$ je Nashevo ravnovesje $\iff$ $\p,\q$ maxmin strategiji.

\end{izrek}
\vspace{0.5cm}

\begin{posledica}

Za matrične igre je $\set{(\p,\q)\in\Pi_1\times\Pi_2; ~(\p,\q)~\text{je Nashevo ravnovesje}}$ konveksna.

\end{posledica}
\vspace{0.5cm}

% *************************************************************************************************

\subsection{Posebne matrične igre}
\vspace{0.5cm}

\begin{definicija}

Naj bo $\AA$ matrična igra, $\AA = (a_{ij})_{i\in [m], j \in [n]}$. Položaj $(i,j)$ je \textit{sedlo}, če velja:
\begin{align*}
\forall i' \in [m]:& ~a_{ij} \geq a_{i'j} \\
\forall j' \in [n]:& ~a_{ij} \leq a_{ij'}.
\end{align*}

\end{definicija}
\vspace{0.5cm}

\begin{trditev}

Če ima matrična igra $\AA$ sedlo na $(i,j)$, potem:
\begin{itemize}
	\item $V(a) = a_{ij}$;
	\item $(\delta(i),\delta(j))$ je Nashevo ravnovesje;
	\item $\delta(i)$ je maxmin za 1. igralca;
	\item $\delta(j)$ je maxmin za 2. igralca.
\end{itemize}

\end{trditev}
\vspace{0.5cm}

\begin{opomba}

Ko obstaja sedlo, je lahko maxmin strategij več.

\end{opomba}
\vspace{0.5cm}

% *************************************************************************************************

\pagebreak

% #################################################################################################

\section{Bayesove igre}
\vspace{0.5cm}

\begin{definicija}

\textit{Bayesova igra} je $7$-terica $\oklepaj{N, \Omega, (p_i)_{i \in N}, (A_i)_{i \in N}, (u_i)_{i \in \N}, (T_i)_{i \in \N}, (\tau_i)_{i \in N}}$, kjer:
\begin{itemize}
	\item $N$ ... končna, neprazna množica igralcev;
	\item $\Omega$ ... neprazna množica stanj; posamezno stanje označimo $\omega$;
	\item $p_i: \Omega \rightarrow [0,1]$ ... funkcija verjetnosti za stanja igralca $i$ (``predhodno prepričanje'');
	\item $A_i$ ... neprazna, končna množica akcija za igralca $i \in N$; profil akcij označimo s $\A = \prod_{i \in N} A_i$;
	\item $u_i: \Omega \times \A \rightarrow \R$ ... funkcija koristnosti za igralca $i$;
	\item $T_i$ ... neprazna množica signalov za igralca $i$;
	\item $\tau_i$ ... signalna funkcija za igralca $i$.
\end{itemize}

\end{definicija}
\vspace{0.5cm}

\begin{definicija}

Vsako stanje $\omega \in \Omega$ v Bayesovi igri $\Gamma$ določi strateško igro s funkcijami koristnosti $\Gamma_\omega = \oklepaj{N, (A_i)_{i \in N}, (u_i(\omega,\cdot))_{i \in N}}$.

\end{definicija}
\vspace{0.5cm}

\begin{opomba}

Posamezen $i \in N$ ima v vsaki $\Gamma_\omega$ na voljo iste akcije.

\end{opomba}
\vspace{0.5cm}

\begin{definicija}

Definiramo
\begin{align*}
\phi_{\bay\rightarrow\kor}: \set{\text{Bayesove igre}} ~&\rightarrow~ \set{\text{igre s funkcijami koristnosti}} \\
(N,\Omega,p_i,A_i,u_i,T_i,\tau_i) ~&\mapsto~ (\mathcal{T}, (A_{(i,t_i)}=A_i)_{(i,t_i)\in\mathcal{T}}, (\tilde{u}_{(i,t_i)})_{(i,t_i)\in\mathcal{T}})
\end{align*}
kjer je $\mathcal{T}$ množica tipov. Bayesovo ravnovesje v Bayesovi igri $\Gamma$ $\equiv$ Nashevo ravnovesje v $\phi_{\bay\rightarrow\kor}(\Gamma)$ (čisto ali mešano). Bayesovo ravnovesje vedno obstaja, le so $A_i$ končne $\forall i \in N$.

\end{definicija}
\vspace{0.5cm}

% *************************************************************************************************

\pagebreak

% #################################################################################################

\section{Ekstenzivne igre}
\vspace{0.5cm}

\begin{definicija}

\textit{Ekstenzivna igra} je $4$-terica $(N, \TT, \P, (u_i)_{i \in N})$, kjer:
\begin{itemize}
	\item $N$ ... končna množica igralcev
	\item $\TT$ ... drevo s korenom $r$ (brez neskončno poti)
	\begin{itemize}
		\item $\mathscr{L}(\TT)$ ... množica listov
		\item $\mathscr{V}(\TT)$ ... množica vozlišč
		\item $\mathscr{E}(\TT)$ ... množica povezav
		\item $\mathscr{E}_v$ ... množica povezav od $v$ navzdol
	\end{itemize}
	\item $\P: \V(\TT)\setminus\L(\TT) \rightarrow N$ ... določi, kdo je na vrsti
	\item $u_i: \L(\TT) \rightarrow \R$ je funkcija preferenc igralca $i$ na $\L(\TT)$
\end{itemize}

\end{definicija}
\vspace{0.5cm}

\begin{opomba}
~
\begin{itemize}
	\item Če ima $\TT$ neskončne poti, potem se igra na konča.
	\item $u_i$ so lahko tudi funkcije koristnosti.
	\item Večkrat je ``ime'' vozlišča zgodovina akcij, da pridemo do tega stanja.
	\item Igra poteka od korena navzdol.
	\item Vsaka ekstenzivna igra $\Gamma = (N, \TT, \P, (u_i)_{i \in N})$ določi za vsako vozlišče $v \in \V(\TT)$ podigro $\Gamma(v) = (N, \TT(v), \P\big|_{\V(\TT(v))\setminus\L(\TT(v))}, (u_i\big|_{\L(\TT(v))})_{i \in N})$
\end{itemize}

\end{opomba}
\vspace{0.5cm}

\begin{definicija}

Definiramo preslikavo \textit{izid}:
\begin{align*}
\O: \prod_{i \in N}S_i = \S ~&\rightarrow~ \L(\TT) \\
(s_i)_{i \in N} ~&\mapsto~ \text{list drevesa, kjer se konča igra, če igralci uporabljajo strategije}~ (S_i)_{i \in N},
\end{align*}
torej dobimo edini list, za katerega obstaja pot v drevesu od korena, ki uporabi samo povezave v $\bigcup_{i \in N} S_i$.

\end{definicija}
\vspace{0.5cm}

\begin{definicija}

Definiramo
\begin{align*}
\phi_{\eks\rightarrow\pref}: ~\set{\text{ekstenzivne igre}} ~&\rightarrow~ \set{\text{igre s funkcijami preferenc}} \\
(N, \TT, \P, u_i) ~&\mapsto~ (N, S_i, (u_i \circ \O)_{i \in N})
\end{align*}
Nasheva ravnovesja v ekstenzivni igri $\Gamma$ so Nasheva ravnovesja v strateški igri $\phi_{\eks\rightarrow\pref}(\Gamma)$.

\end{definicija}
\vspace{0.5cm}

\begin{definicija}

Definiramo funkcija \textit{izida za vozlišče $v$}:
\begin{align*}
\O_v: ~\S ~&\rightarrow~ \L(\TT) \\
(s_i)_{i \in N} ~&\mapsto~ \text{list v}~\TT(v),~\text{kjer se konča igra, če začnemo v vozlišču}~v
\end{align*}

\end{definicija}
\vspace{0.5cm}

\begin{definicija}

Profil strategij $\mathbf{s} = (s_i)_{i \in N}$ je vgnezdeno Nashevo ravnovesje, če velja
$$\forall v \in \V(\TT),~\forall i \in N,~\forall s'_i \in \mathbf{s}:~(u_i \circ \O_v)(\mathbf{s}) ~\geq~ (u_i \circ \O_v)(\mathbf{s} \mid s'_i)$$
oz. $(u_i \circ \O_v)(s_1,\ldots,s_n) \geq (u_i \circ \O_v)(s_1,\ldots,s_{i-1},s'_i,s_{i+1},\ldots,s_n)$.

\end{definicija}
\vspace{0.5cm}

\begin{definicija}

Ekstenzivna igra z nepopolno informacijo vsebuje:
\begin{itemize}
	\item $N$ ... neprazna množica igralcev
	\item $\TT$ ... drevo s korenom brez neskončnih poti
	\begin{itemize}
		\item $\L(\TT)$ ... listi
		\item $\E_v$ ... povezave iz $v$ navzdol
	\end{itemize}
	\item $u_i: \L(\TT) \rightarrow \R$ funkcija koristnosti
	\item $I_1,\ldots,I_n$ ... razdelitev $\V(\TT)\setminus\L(\TT)$. Velja:
	$$\forall j,~\forall u,v \in I_j: ~|\E_u| = |\E_v|$$
	\item $\P: \set{I_1,\ldots,I_n} \rightarrow \N \cup \set{\text{slučaj}}$ ... določi, kdo je na vrsti na vsaki informacijski množici
	\item Za vsako informacijsko množico $I_j$ naredimo indentifikacije med povezavami v $\bigcup_{u \in I_i} \E_u$
	\item Na informacijski množici, kjer je slučaj na vrsti, funkcija verjetnosti na povezavah
\end{itemize}

Podigre morajo vsebovati bodisi celo informacijsko množico ali nič iz informacijske množice. $\Gamma(v)$ je podigra, če velja:
$$\forall I_j: ~\V(\TT(v)) \supseteq I_j \quad\text{ali}\quad \V(\TT(v)) \cap I_j = \emptyset.$$

\end{definicija}
\vspace{0.5cm}

\pagebreak

% #################################################################################################

\section{Kooperativne igre}
\vspace{0.5cm}

% *************************************************************************************************

\subsection{Nashev produkt}
\vspace{0.5cm}

\begin{trditev}

Naj bo $K$ konveksna in kompaktna množica v $\R^2$. Predpostavimo, da $\exists (x,y) \in K: x>x_0$, $y>y_0$. Potem obstaja enolična točka $(x^*,y^*)$ znotraj $K$, $x^* \geq x_0$, $y^* \geq y_0$ in
$$\Phi(K,x_0,y_0) ~=~ (x^*-x_0)(y^*-y_0),$$
kjer je $\Phi(K,x_0,y_0):=\sup(x-x_0)(y-y_0)$. Definiramo to enolično točko kot $\N(K,x_0,y_0)$.

\end{trditev}
\vspace{0.5cm}

\begin{trditev}

Naj bo $\TT$ trikotnik z oglišči $(0,0),(\alpha,0),(0,\beta)$, $\alpha,\beta>0$. Potem
$$\N(\TT,0,0) ~=~ \oklepaj{\frac{\alpha}{2},\frac{\beta}{2}}.$$

\end{trditev}
\vspace{0.5cm}

\begin{trditev}

Naj bosta $f_1(x) = \alpha_1 x + \beta_1$, $\alpha_1$, $f_2(y) = \alpha_2 y + \beta_2$, $\alpha_2 > 0$. Za vsak $K$ definiramo $K_f = \set{\oklepaj{f_1(x),f_2(x)} \mid (x,y)\in K}$. Potem velja
$$\N\oklepaj{K_f, f_1(x_0), f_2(y_0)} ~=~ (f_1,f_2)\oklepaj{\N(K,x_0,y_0)}.$$

\end{trditev}
\vspace{0.5cm}

% *************************************************************************************************

\subsection{Nashev model pogajanja}
\vspace{0.5cm}

\begin{definicija}
~
\begin{itemize}
	\item 2 igralca
	\item $\D$ ... dopustni izidi sporazumov, konveksen
	\item $(a_0,b_0)\in\R^2$ ... točka nesporazuma (\textit{status quo})
\end{itemize}
Če do sporazuma ne pride, potem bo $(a_0,b_0)$ izid igre. Predpostavimo:
$$\exists (a,b) \in \D: ~a>a_0,~b>b_0$$
Iščemo funkcijo 
$$\varphi(\D,a_0,b_0) ~=~ \oklepaj{\varphi_1(\D,a_0,b_0),\varphi_2(\D,a_0,b_0)} \in \R^2,$$
ki bo ``rešitev''. Lastnosti funkcije $\varphi$:
\begin{enumerate}

\item[(A1)] (Dopustnost):
$$\varphi(\D,a_0,b_0) \in \D$$

\item[(A2)] (Individualno racionalnost):
$$\varphi_1(\D,a_0,b_0) \geq a_0,~\varphi_2(\D,a_0,b_0)\geq b_0$$

\item[(A3)] (Paretova optimalnost):
$$\forall (a,b)\in\D,~(a,b)\neq\varphi(\D,a_0,b_0):~a<\varphi_1(\D,a_0,b_0),~b<\varphi_2(\D,a_0,b_0).$$

\item[(A4)] (Simetrija): Če je $\D$ simetrična glede na premico $a=b$ \hbox{($(a,b)\in\D \iff (b,a)\in\D$)}, potem
$$\varphi_1(\D,0,0) ~=~ \varphi_2(\D,0,0)$$

\item[(A5)] (Preoblikovanje koristnosti): Za poljubni funkciji $f_1(x) = \alpha_1 x + \beta_1$, $f_2(y) = \alpha_2 y + \beta_2$, $\alpha_1,\alpha_2>0$ velja
$$\oklepaj{f_1\oklepaj{\varphi_1(\D,a_0,b_0)}, f_2\oklepaj{\varphi(\D,a_0,b_0)}} ~=~ \varphi\oklepaj{\D_f,f_1(a_0),f_2(b_0)}$$

\item[(A6)] (Neodvisnost od irelevantnih možnosti)
$$\D' \subseteq \D, \varphi(\D,a_0,b_0)\in\D' ~\Rightarrow~ \varphi(\D',a_0,b_0) = \varphi(\D,a_0,b_0)$$

\end{enumerate}

\end{definicija}
\vspace{0.5cm}

\begin{izrek}[Nash]

Obstaja enolična funkcija $\varphi$, ki zadošča lastnostim (A1)-(A6) in sicer
$$\varphi(\D,a_0,b_0) ~=~ \N(\D,a_0,b_0).$$

\end{izrek}
\vspace{0.5cm}

\begin{posledica}

$[A,B]$ bimatrična igra, prenosljiva koristnost. $\D := \D_{PK}(A,B)$
$$\varphi(\D,a_0,b_0) ~=~ \N(\D,a_0,b_0) ~=~ \oklepaj{\frac{a_0-b_0+\sigma}{2},\frac{b_0-a_0+\sigma}{2}},$$
kjer je $\sigma = \max_{i,j}(a_{ij} + b_{ij})$.

\end{posledica}
\vspace{0.5cm}

% *************************************************************************************************

\subsection{Koalicijske igre}
\vspace{0.5cm}

\begin{definicija}

Koaliciska igra je par $(N,v)$, pri čemer je:
\begin{itemize}
	\item $N$ ... množica igralcev
	\item $v: 2^N \rightarrow \R$, $v(\emptyset)=0$ ... karakteristična funkcija
\end{itemize}

\end{definicija}
\vspace{0.5cm}

\begin{definicija}

Naj bo $S \subseteq N$ koalicija. Definiramo:
\begin{itemize}
	\item \textit{Superaditivnost}:
	$$\forall S,T \subseteq M, ~S \cap T = \emptyset: ~v(S)+v(T) \leq v(S\cup T)$$
	\item \textit{Subaditivnost}:
	$$\forall S,T \subseteq M, ~S \cap T = \emptyset: ~v(S)+v(T) \geq v(S\cup T)$$
\end{itemize}

\end{definicija}
\vspace{0.5cm}

\begin{definicija}

Naj bo $\Gamma = (N,v)$ koalicijska igra:
\begin{itemize}
	\item vektor plačil: $(x_i)_{i \in N} \in \R^N$
	\item vektor plačil je skupno racionalni, če
	$$\sum_{i \in N} x_i ~=~ v(N)$$
	\item vektor plačil je individualno racionalen, če
	$$\forall i \in N: ~x_i \geq v\oklepaj{\set{i}}$$
	\item \textit{imputacija} je plačilni vektor, ki je skupno in individualno racionalen. Množica imputacij:
	$$I(\Gamma) ~=~ \big\{(x_i)_{i \in N}\in\R^N \mid \forall i \in N: ~x_i \geq v(\set{i}),~\sum_{i \in N}x_i = v(N)\big\}$$
	\item \textit{jedro} koalicijske igre $\Gamma = (N,v)$ je
	$$J(\Gamma) ~=~ \big\{(x_i)_{i \in N}\in\R^N \mid \sum_{i \in N}x_i = v(N),~\forall S \subset N: \sum_{i \in S}x_i \geq v(S)\big\}$$
	\item imptuacija $\x = (x_i)_{i \in N}$ je stabilna preko koalicije $S$, če velja
	$$\sum_{i \in S} x_i \geq v(s)$$
	\item jedro je množica impuzacij, ki so skupno racionalne in stabilne preko vseh koalicij. Velja $J(\Gamma) \subseteq I(\Gamma)$.
\end{itemize}

\end{definicija}
\vspace{0.5cm}

\begin{opomba}

Če je $\Gamma$ superaditivna, potem $I(\Gamma) \neq \emptyset$.

\end{opomba}
\vspace{0.5cm}

\begin{definicija}[Shapleyeve vrednosti]

Naj bo $N = [n]$. Iščemo funkcijo $\phi = (\phi_1,\ldots,\phi_n): \set{v:2^{[n]}\rightarrow\R}\rightarrow\R^n$, $v(\emptyset)=0$, ki ima naslednje lastnosti:
\begin{enumerate}

\item[(A1)] (Učinkovitost/skupna racionalnost):
$$\forall v: ~\sum_{i \in [n]} \phi_i(v) ~=~ v([n])$$

\item[(A2)] (Simetrija):
$$\forall i,j \in [n],~\forall v: ~\oklepaj{\oklepaj{\forall S \subseteq [n]\setminus\set{i,j}: v(S \cup \set{i}) = v(S \cup \set{j})} ~\Longrightarrow~ \phi_i(v) = \phi_j(v)}$$

\item[(A3)] (Dummy player):
$$\forall i \in [n],~\forall v: ~\oklepaj{\oklepaj{\forall S \subseteq [n]\setminus\set{i}: v(S) = v(S \cup \set{i})} \Longrightarrow \phi_i(v) = 0}$$

\item[(A4)] (Aditivnost):
$$\forall i \in [n],~\forall v,v': ~\phi_i(v+v') = \phi_i(v) + \phi_i(v')$$

\end{enumerate}
Obstaja enolično določena funkcija $\phi$, ki zadošča pogojem (A1)-(A4).

\end{definicija}
\vspace{0.5cm}

\begin{trditev}

Naj bo $\pi: [n] \rightarrow [n]$ permutacija. 
$$N(\pi,i) = \set{j \in [n] \mid \pi^{-1}(j) \leq \pi^{-1}(i)}$$
je množica igralcev, ki pridejo pred $i$ ter $i$. Naj bo
$$\Delta(\pi,i,v) ~=~ v\oklepaj{N(\pi,i)} - v\oklepaj{N(\pi,i)\setminus\set{i}}.$$
Definiramo
\begin{align*}
\phi_i(v) ~&=~ \mathbb{E}_\pi[\Delta(\pi,i,v)] \\
&=~ \frac{1}{n!} \sum_{\pi \in S_n} \Delta(\pi,i,v)) \\
&=~ \frac{1}{n!} \sum_{S \subseteq N,~i \notin S} |S|!(n-|S|-1)!(v(S \cup \set{I}) - v(S))
\end{align*}
Te $(\phi_i)_{i \in N}$ so Shapleyeve vrednosti in ustrezajo pogojem (A1)-(A4).

\end{trditev}
\vspace{0.5cm}

\begin{izrek}

$(\phi_i)_{i \in N}$ je enolična funkcija, ki zadošča aksiomom (A1)-(A4).

\end{izrek}
\vspace{0.5cm}

% *************************************************************************************************

\pagebreak

% #################################################################################################

\end{document}