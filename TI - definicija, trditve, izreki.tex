\documentclass[11pt]{article}
\usepackage[utf8]{inputenc}
\usepackage[slovene]{babel}

\usepackage{amsthm}
\usepackage{amsmath, amssymb, amsfonts}
\usepackage{relsize}
\usepackage{mathrsfs}

\newcommand{\R}{\mathbb{R}}
\newcommand{\N}{\mathbb{N}}
\newcommand{\A}{\mathcal{A}}
\newcommand{\diff}{\overset{\text{def}}{\iff}}
\newcommand{\set}[1]{\{#1\}}
\newcommand{\oklepaj}[1]{\left(#1\right)}
\newcommand{\oglati}[1]{\left[#1\right]}

\theoremstyle{definition}
\newtheorem{definicija}{Definicija}[section]

\theoremstyle{definition}
\newtheorem{trditev}{Trditev}[section]

\theoremstyle{definition}
\newtheorem{izrek}{Izrek}[section]

\theoremstyle{definition}
\newtheorem{metoda}{Metoda}[section]

\newtheorem*{posledica}{Posledica}
\newtheorem*{opomba}{Opomba}
\newtheorem*{komentar}{Komentar}
\newtheorem{lema}{Lema}
\newtheorem*{notacija}{Notacija}
\newtheorem*{dokaz}{Dokaz}
\newtheorem*{posplošitev}{Posplošitev}
\newtheorem*{dogovor}{Dogovor}
\newtheorem*{sklep}{Sklep}

\title{Teorija iger - definicije, trditve in izreki}
\author{Oskar Vavtar \\
po predavanjih profesorja Sergia Cabella}
\date{2021/22}

\begin{document}
\maketitle
\pagebreak
\tableofcontents
\pagebreak

% #################################################################################################

\section{Strateške igre s funkcijami preferenc}
\vspace{0.5cm}

% *************************************************************************************************

\subsection{Uvod}
\vspace{0.5cm}

\begin{definicija}

Naj bo $\A$ množica. \textit{Funkcija preferenc} na množici $\A$ je preslikava $u: \A \rightarrow \R$ funkcija preferenc. Intuicija: $\forall a,a' \in \A, a \neq a'$:
\begin{itemize}
	\item $u(a)>u(a')$ ``$\iff$'' $a$ je boljše kot $a'$
	\item $u(a)<u(a')$ ``$\iff$'' $a$ je slabše kot $a'$
	\item $u(a)=u(a')$ ``$\iff$'' med $a$ in $a'$ smo indiferentni
\end{itemize}

\end{definicija}
\vspace{0.5cm}

\begin{opomba}
~
\begin{itemize}
	\item Različne funkcije lahko določijo iste preference.
	\item Obravnavali bomo tudi več funkcij preferenc na isti množici (vsak igralec ima lahko svojo funkcijo).
	\item Preverence določimo kvalitativno, ne kvantitativno - pomemben je le vrstni red, same vrednosti ne.
	\item Namesto $\R$ bi lahko uporabili poljubno drugo linearno urejeno množico.
\end{itemize}

\end{opomba} 
\vspace{0.5cm}

\begin{definicija}

\textit{Strateška igra s funkcijami preferenc} je trojica
$$(N, (A_i)_{i \in N}, (u_i)_{i \in N}),$$
pri čemer:
\begin{itemize}
	\item $N$ je končna množica \textit{igralcev}.
	\item Za vsakega igralca $i \in N$ je $A_i$ neprazna množica \textit{akcij} za $i \in N$. Naj bo 
	$$\A ~:=~ \prod_{i \in N} A_i$$
	množica \textit{profilov akcij}.
	\item Za vsakega igralca $i \in N$ je $u_i: \A \rightarrow \R$ je funkcija preferenc na $\A$ za igralca $i$.
\end{itemize}

\end{definicija}
\vspace{0.5cm}

\begin{opomba}

Ponavadi: $N = [n] = \set{1,\ldots,n}$. V tem primeru imamo trojico
$$([n],(A_1,\ldots,A_n),(u_1,\ldots,u_n)),$$
$\A=A_1 \times \ldots \times A_n$ ter $u_i: A_1 \times \ldots \times A_n \rightarrow \R$.

\end{opomba}
\vspace{0.5cm}

% *************************************************************************************************

\subsection{Čisto Nashevo ravnotežje}
\vspace{0.5cm}

\begin{notacija}

$$(x_\alpha, x_\beta, x_\gamma \mid y; \beta) ~=~ (x_{-\beta}, y) ~=~ (x_\alpha, y, x_\gamma)$$
Za funkcije preferenc:
$$u_i(x_1,\ldots,x_m \mid y) ~=~ u_i(x_1,\ldots,x_{i-1},y,x_{i+1},\ldots,x_n).$$

\end{notacija}
\vspace{0.5cm}

\begin{definicija}

Naj bo $\Gamma = (N, (A_i)_{i \in N}, (u_i)_{i \in N})$ strateška igra s funkcijami preferenc. Naj bo
$$\A ~=~ \prod_{i \in N} A_i.$$
Profil akcij $a^* \in \A$ je \textit{čisto Nashevo ravnovesje} $\diff$
$$\forall i \in N, ~\forall b \in A_i: ~u_i(a^*) \geq u_i(a^* \mid b).$$
Tak $a^* \in \A$ je \textit{strogo čisto Nashevo ravnovesje} $\diff$
$$\forall i \in N, ~\forall b \in A_i\setminus\set{a_i^*}: ~u_i(a^*) > u_i(a^* \mid b).$$

\end{definicija}
\vspace{0.5cm}

\pagebreak

\begin{definicija}

Naj bo $\Gamma = (N, (A_i)_{i \in N}, (u_i)_{i \in N})$ strateška igra s funkcijami preferenc. Označimo
$$\A ~=~ \prod_{n \in N} A_i.$$
\textit{Najboljši odgovor} igralca $i \in N$ je 
\begin{align*}
B_i: ~\A ~&\rightarrow~ 2^{A_i} ~=~ \set{B \mid B \subseteq A_i} \\
a ~&\mapsto~ \set{b \in A_i \mid \forall c \in A_i: u_i(a \mid b) \geq u_i(a \mid c)} \\
&=~ \set{b \in A_i \mid u_i(a \mid b) ~=~ \max_{c \in A_i}(a \mid c)}.
\end{align*}
Za $N = [n] = \set{1,\ldots,n}$ je formula v definiciji za igralca $i=1:$
$$B_1(a_1,\ldots,a_n) ~=~ \set{b \in A_1 \mid u_1(b,a_2,\ldots,a_n) = \max_{c \in A_1} u_1(c,a_2,\ldots,a_n)},$$
analogno za $i=2,\ldots,n$.

\end{definicija}
\vspace{0.5cm}

\begin{opomba}
~
\begin{itemize}
	\item Bolj pravilno bi bilo reči ``\textit{množica najboljših odgovorov}''.
	\item Večkrat velja $|B_i(a)|=1$ (le en najboljši odgovor). V tem primeru pišemo brez $\set$.
	\item Pri definiciji $B_i$ nima $a_i$ nobene vloge. Za dva igralca bomo ponavadi napisali
	\begin{align*}
	B_1(a_2) ~&\equiv~ B_1(*,a_2) \\
	B_2(a_1) ~&\equiv~ B_2(a_1,*).
	\end{align*}
\end{itemize}

\end{opomba} 
\vspace{0.5cm}

\begin{trditev}

Profil akcij $a^* = (a_i^*)_{i \in N}$ je čisto Nashevo ravnovesje $\iff$
$$\forall i \in N: ~a_i^* \in B_i(a^*).$$

\end{trditev}
\vspace{0.5cm}

\pagebreak

\begin{trditev}
Profil akcij $a^* = (a_i^*)_{i \in N}$ je strogo čisto Nashevo \hbox{ravnovesje $\iff$}
$$\forall i \in N: ~a_i^* \in B_i^\text{str}(a^*),$$
kjer $B_i^\text{str}$ definiramo kot
\begin{align*}
B_i^\text{str}: ~\A ~&\rightarrow~ 2^{A_i} \\
a ~&\mapsto~ \set{b \in A_i \mid \forall c \in A_i \setminus \set{b}: u_i(a^* \mid b) > u_i(a^* \mid c)} \\
&=~ \begin{cases}
\text{edini max}; ~&\text{če obstaja}, \\
\emptyset; ~&\text{sicer}.
\end{cases}
\end{align*}
\end{trditev}
\vspace{0.5cm}

% *************************************************************************************************

\subsection{Dominacije}
\vspace{0.5cm}

\begin{definicija}

Naj bo $\Gamma = (N, (A_i)_{i \in N}, (u_i)_{i \in N})$ strateška igra s funkcijo preferenc. Označimo
$$\A ~=~ \prod_{i \in N} A_i.$$
Akcija $b \in A_i$ \textit{šibko dominira} akcijo $c \in A_i$, če velja
$$\forall a \in \A: ~u_i(a \mid b) \geq u_i(a \mid c).$$
Akcija $b \in A_i$ \textit{strogo dominira} akcijo $c \in A_i$, če velja
$$\forall a \in \A: ~u_i(a \mid b) > u_i(a \mid c).$$

\end{definicija}
\vspace{0.5cm}

\begin{trditev}

Če $b \in A_i$ strogo dominira $c \in A_i$, potem ne obstaja čisto Nashevo ravnovesje $a^* = (a_i^*)_{i \in N}$ z $a_i^* = c$. Če $b \in A_i$ dominira $c \in A_i$, potem obstaja strogo čisto Nashevo ravnovesje $a^* = (a_i^*)_{i \in N}$ z $a_i^* = c$.

\end{trditev}
\vspace{0.5cm}

% *************************************************************************************************

\pagebreak

% #################################################################################################

\section{Strateške igre s funkcijami koristnosti}
\vspace{0.5cm}

% *************************************************************************************************

\subsection{Uvod}
\vspace{0.5cm}

\begin{definicija}

Naj bo $A = (a_1,\ldots,a_\Pi)$ končna množica in $\pi$ funkcija verjetnosti:
$$\pi ~\sim~ \begin{pmatrix}
a_1 & \ldots & a_\Pi \\
\pi(a_1) & \ldots & \pi(a_\Pi)
\end{pmatrix}.$$ 
Množica $\pi(A)$, definirana kot
$$\pi(A) ~=~ \set{((\pi(a_i))_{a_i \in A} \mid \forall a_i \in A_i: \pi(a_i) \geq 0, ~\sum_{a_i \in A} \pi(a_i) = 1}$$
je \textit{množica loterij} na $A$.

\end{definicija}
\vspace{0.5cm}

\begin{definicija}

Naj bo $A$ končna. \textit{Funkcija koristnosti} na $A$ je prelikava $u: A \rightarrow \R$, ki določa preference na množici $\pi(A)$ in sicer
\begin{align*}
\hat{u}: \pi(A) ~&\rightarrow~ \R \\
\pi = (\pi(a))_{a \in A} ~&\mapsto~ \sum_{a \in A} \pi(a) u(a),
\end{align*}
če je $\hat{u}$ razširitev funkcije $u$. Osnovni princip bo:
$$\hat{u}(\pi) \geq \hat{u}(\pi') ~~~\iff~~~ \pi ~\text{ni slabše od}~ \pi'.$$

\end{definicija}
\vspace{0.5cm}

\begin{opomba}

Lahko imamo različni funkciji koristnosti, ki določata iste preference na $\pi(A)$. En del teorije koristnosti se ukvaraja z obratno smerjo, in sicer za katere preference nad $\pi(A)$ obstaja funkcija koristnosti $u: A \rightarrow \R$, za katero imamo razširitev iste preference $\pi(A)$.

\end{opomba}
\vspace{0.5cm}

% *************************************************************************************************

\subsection{Igre koristnosti}
\vspace{0.5cm}

\begin{definicija}

\textit{Strateška igra s funkcijami koristnosti} je trojica 
$$\Gamma ~=~ (N, (A_i)_{i \in N}, (u_i)_{i \in N}),$$
pri čemer:
\begin{itemize}
	\item je $N$ neprazna in končna množica igralcev.
	\item je $A_i$ neprazna in končna množica akcij za igralca $i \in N$. Naj bo $A = \prod_{i \in N} A_i$ množica profilov akcij.
	\item je $u_i: A \rightarrow \R$ funkcija koristnosti za igralca $i \in N$.
\end{itemize}

\noindent Množica strategij za igralca $i \in N$ je $S_i = \pi(A_i)$. Strategija $\pi \in S_i$ meša akcije $\set{a \in A_i \mid \pi_i(a) > 0}$. Strategija $\pi \in S_i$ je čista, če je $\pi = S(a)$ za nek $a \in A_i$. Množica profilov strategij je $S = \prod_{i \in N} S_i$.

\end{definicija}
\vspace{0.5cm}

\begin{trditev}

Definiramo:
\begin{align*}
u_i: S = \prod_{j \in N} S_j ~&\rightarrow~ \R \\
\pi = (\pi_j)_{j \in N} ~&\mapsto~ \sum_{a \in A} \oklepaj{\prod_{j \in N}\pi_j(a_j)} u_i(a).
\end{align*}
Velja:
$$u_i(\pi) ~=~ \sum_{a_i \in A} \pi_i(a_i) \oglati{u_i(\pi \mid \delta(a_i)}.$$

\end{trditev}
\vspace{0.5cm}

% *************************************************************************************************

\subsection{Nasheva ravnovesja}
\vspace{0.5cm}

\begin{definicija}

Naj bo $\Gamma(N, (A_i)_{i \in N}, (u_i)_{i \in N})$ strateška igra s funkcijami koristnosti. Profil strategij 
$$\Pi^* = (\pi_i^*) ~\in~ S = \prod_{i \in N} S_i$$
je Nashevo ravnovesje $\diff$
$$\forall i \in N, ~\forall \pi'_i \in S_i, ~\pi'_i \neq \pi_i^*: ~u_i^*(\pi^*) ~>~ u_i(\pi^* \mid \pi'_i).$$
Imamo preslikavo
\begin{align*}
\phi_{\mathscr{K} \rightarrow \mathscr{P}}: \set{\text{igra s funkcijami koristnosti}} ~&\rightarrow~ \set{\text{igra s funkcijami preferenc}} \\
(N, (A_i)_{i \in N}, (u_i)_{i \in N}) ~&\mapsto~ (N, (S_i = \pi(A_i))_{i \in N}, (u_i)_{i \in N}).
\end{align*}
$\phi_{\mathscr{K} \rightarrow \mathscr{P}}$ je \textit{mešana razširitev} igre $\Gamma$.

\end{definicija}
\vspace{0.5cm}

\begin{trditev}

Naj bo $\Gamma$ strateška igra s funkcijami koristnosti in naj bo $\pi$ profil strategij v $\Gamma$. $\pi$ je Nashevo ravnovesje v $\Gamma$ $\iff$ $\pi$ je čisto Nashevo ravnovesje v $\phi_{\mathscr{K} \rightarrow \mathscr{P}}(\Gamma)$.

\end{trditev}
\vspace{0.5cm}

\begin{izrek}[Nash, 1949]

Vsaka strateška igra s funkcijami koristnosti ima vsaj eno Nashevo ravnovesje.

\end{izrek}
\vspace{0.5cm}

\begin{opomba}
~
\begin{itemize}
	\item Dokaz je nekonstruktiven, nič ne pove kako Nashevo ravnovesje najti.
	\item Pomembno je, da ima vsak igralec končno množico akcij. Obstajajo sicer tudi posplošitve za neskončno mnogo akcij.
	\item Če je igra generična, potem je število Nashevih ravnovesij liho.
\end{itemize}

\end{opomba}
\vspace{0.5cm}

\begin{trditev}

Profil strategij $\pi^* = (\pi_i^*)_{i\in N}$ je Nashevo ravnovesje
$$\iff~ \forall i \in N ~\forall a_i \in A_i: ~u_i(\pi^*) \geq u_i(\pi^* \mid \delta(a_i)).$$

\end{trditev}
\vspace{0.5cm}

\begin{trditev}[Princip indiferentnosti]

Če je $\pi^* = (\pi_i^*)_{i \in N}$ Nashevo ravnovesje, potem velja
$$\forall i \in N ~\forall a_i \in A_i: ~(\pi_i^*(a_i) > 0 ~\Rightarrow~ u_i(\pi^*) = u_i(\pi^* \mid \delta(a))).$$

\end{trditev}
\vspace{0.5cm}

\begin{opomba}
~
\begin{itemize}
	\item Uporabna, ker imamo enačbe.
	\item To ni karakterizacija. Uporabimo za iskanje kandidatov.
	\item Ko ima igralec $i$ $|A_i| = 2$, potem, ko velja
	$$u_i(\pi^* \mid \delta(a_i)) ~=~ u_i(\pi^* \mid \delta(b_i)) ~=~ u_i(\pi^*)$$
	avtomatično velja pogoj sistema neenačb za igralca $i$:
	\begin{align*}
	u_i(\pi^* \mid \delta(a_i)) ~\leq~ u_i(\pi^*) \\
	u_i(\pi^* \mid \delta(b_i)) ~\leq~ u_i(\pi^*)
	\end{align*}
\end{itemize}

\end{opomba}
\vspace{0.5cm}

% *************************************************************************************************

\subsection{Dominacije}
\vspace{0.5cm}

\begin{definicija}

Strategija $\alpha_i \in S_i$ \textit{(šibko) dominira} strategijo $\beta_i \in S_i$, če velja
$$\forall \pi \in S: ~u_i(\pi \mid \alpha_i) ~\geq~ u_i(\pi \mid \beta_i).$$
Strategija $\alpha_i \in S_i$ \textit{strogo dominira} $\beta_i \in S_i$, če velja
$$\forall \pi \in S: ~u_i(\pi \mid \alpha_i) ~>~ u_i(\pi \mid \beta_i).$$

\end{definicija}
\vspace{0.5cm}

\begin{trditev}

Naj bo $\pi^* = (\pi_i^*)_{i \in N}$ Nashevo ravnovesje. Če strategija \hbox{$\alpha_i \in S_i$} strogo dominira $\delta(b_i)$, potem 
$$\pi_i^*(b_i) ~=~ 0.$$

\end{trditev}
\vspace{0.5cm}

% *************************************************************************************************

\pagebreak

% #################################################################################################

\end{document}